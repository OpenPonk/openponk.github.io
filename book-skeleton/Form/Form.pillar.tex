% -*- mode: latex; -*-
\ifx\wholebook\relax\else

	% Lulu:
	\documentclass[a4paper,10pt,twoside]{book}
	\usepackage[
		papersize={6.13in,9.21in},
		hmargin={.75in,.75in},
		vmargin={.75in,1in},
		ignoreheadfoot
	]{geometry}
	\input{../support/latex/common.tex}
	% \input{../support/latex/commonLuaTex.tex}
	\setboolean{lulu}{true}
% --------------------------------------------
% A4:
%	\documentclass[a4paper,11pt,twoside]{book}
%	\input{../support/latex/common.tex}
%	\usepackage{a4wide}
% --------------------------------------------    
\graphicspath{
	{figures/}
	{../figures/}
}

\begin{document}
\fi
\sloppy

\chapter{ Form}
To be able to edit a model through a form you need model's controller to describe what and how should be edited.

Override \ct{DCController$>$$>$buildEditorForm:} and describe all the elements required. The recieved \ct{aForm} argument is an instance of \ct{DCDynamicForm}.


\begin{code}{}
buildEditorForm: aForm
	"base class automatically adds edit field for `name`"
	super buildEditorForm: aForm.

	(aForm addText: 'Alternative name')
		text: self model altName;
		whenTextIsAccepted: [ :newValue | self model altName: newValue ].
\end{code}


All \ct{add*:} methods of \ct{DCDynamicForm} return appropriate \textit{Spec} models, so you have their full API at your disposal.
The argument of \ct{add*:} is a string label that will be displayed alongside the control.

The description of available controls follows.
\section{ \ct{addLabel:}}
Add a single label.

\begin{code}{}
aForm addLabel: aString.
\end{code}

Usually not needed as all the other methods add their own label automatically.
\section{ \ct{addText:}}
Add a multiline text area.

\begin{code}{}
(aForm addText: aString)
	text: aString; "set the value of the input"
	whenTextIsAccepted: [ :newValue | ]. "observed change"
\end{code}

\section{ \ct{addTextInput:}}
Add a single line text field.

\begin{code}{}
(aForm addText: aString)
	text: aString; "set the value"
	whenTextIsAccepted: [ :newValue | ].
\end{code}

\section{ \ct{addCheckbox:}}
Add a single checkbox.


\begin{code}{}
(aForm addCheckbox: aString)
	state: aBoolean; "set the state"
	whenChangedDo: [ :aBool | ]. "observed change"
\end{code}

\section{ \ct{addDroplist:}}
Add a droplist with specified items.


\begin{code}{}
(aForm addDroplist: aString)
	items: #(#opt1 #opt2); "collection of options available in the droplist"
	displayBlock: [ :item | item asString ]; "if items are complex object, specify what should be displayed"
	setSelectedItem: oneOfTheItems; "set specific value"
	whenSelectedItemChanged: [ :newValue | ]. "observed change"
\end{code}


For particularities of the models consult Spec's documentation.



\ifx\wholebook\relax\else
   \bibliographystyle{jurabib}
   \nobibliography{scg}\end{document}
\fi
